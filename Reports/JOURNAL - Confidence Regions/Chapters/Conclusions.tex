\section{Conclusions}\label{Sec:Conclusions}

We present and evaluate a new method of building empirical confidence regions in the Entropy-Complexity plane for the analysis of white noise.
The following proposal consists of two stages:
(1)~the construction of empirical confidence regions obtained through the mapping of points in the latent space formed by the application of the principal components analysis under the reference points,
(2)~the application of a hypothesis test to measure the similarity of new sequences to the reference points.
Sequences of true random samples were used to calculate theses empirical regions.

Experiments with true random samples showed that the presented methodology can represent them with good performance.
Although the present work focuses on the study of short sequences, we were able to capture the random behavior of PRNGs already analyzed in the literature and we verified the robustness of our technique and the correlation structure present in the sequences.
