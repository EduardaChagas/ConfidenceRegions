\documentclass[sts]{imsart}

%\usepackage{amsthm,amsmath,natbib}
%\RequirePackage[colorlinks,citecolor=blue,urlcolor=blue]{hyperref}

% put your definitions there:
%\startlocaldefs
%\endlocaldefs

\usepackage[author-year]{amsrefs}
\usepackage{bm,bbm, amsmath}
\usepackage{siunitx}
\usepackage{verbatim}

%\usepackage{amssymb}
\usepackage{enumerate}
\usepackage{color}

\usepackage[caption=false]{subfig}
\usepackage{graphicx}
\graphicspath{{./Figures/}}
\usepackage{booktabs}
\usepackage{float}
\usepackage[figuresright]{rotating}
\usepackage{todonotes}


\usepackage{comment}   
\DeclareSIUnit{\minusalphapercent}{(1-\alpha)\%}
\DeclareSIUnit{\alphapercent}{\alpha\%}

\begin{document}

\begin{frontmatter}

\title{Confidence Regions for Information-Theoretic Descriptors of Time Series}
\runtitle{Confidence Regions}

\author{\fnms{Eduarda T.\ C.} \snm{Chagas}\ead[label=e1]{eduarda.chagas@dcc.ufmg.br}}
\and
\author{\fnms{Marcelo} \snm{Queiroz}\corref{}\ead[label=e2]{marceloqao@gmail.com}}
\and\\
\author{\fnms{Osvaldo A.} \snm{Rosso}\ead[label=e3]{oarosso@if.ufal.br}}
\and
\author{\fnms{Heitor S.} \snm{Ramos}\ead[label=e4]{ramosh@dcc.ufmg.br}}
\and\\
\author{\fnms{Christopher G.\ S.} \snm{Freitas}\ead[label=e5]{christopher@laccan.ufal.br}}
\and
\author{\fnms{Leonardo V.} \snm{Pereira}\ead[label=e6]{lviana@ic.ufal.br}}
\and\\
\author{\fnms{Alejandro C.} \snm{Frery}\ead[label=e7]{alejandro.frery@vuw.ac.nz}}
\address{Victoria University of Wellington\\
PO Box 600\\
Wellington 6140, New Zealand \printead{e7}}
\affiliation{School of Mathematics and Statistics\\
Victoria University of Wellington\\
New Zealand}

\runauthor{E.\ Chagas}

\begin{abstract}
The\citeauthor{PermutationEntropyBandtPompe} methodology has been used successfully in the analysis of time series.
It consists of computing information theory descriptors, using a histogram of ordinal patterns, which are found in a 2D variety: the Entropy-Complexity plane.
So far, the analysis of the dynamics underlying the time series has been carried out from two reference points: those corresponding to a deterministic time series and a series of white noise.
In this article, we provide a first proposal for the construction of empirical confidence regions in the Entropy-Complexity plane for white noise models and we use these regions to verify whether we can capture the randomness of PRNGs in short sequences.
The proposed methodology showed consistency and coherence in its results, managing to discriminate sequences of true random samples, capturing the randomness of generators previously analyzed in the literature and proving to be robust the addition of correlation structures.
\end{abstract}

\begin{keyword}[class=MSC]
\kwd[Primary ]{37M10}
\kwd[; secondary ]{68Q30}
\end{keyword}

\begin{keyword}
\kwd{Random number generators}
\kwd{Bandt–Pompe approach}
\kwd{Entropy-complexity plane}
\kwd{Information Theory}
\end{keyword}

\end{frontmatter}

%\tableofcontents

\section{Introduction}\label{Sec:Intro}

Time Series carry valuable information about the system which produces the data.
Their analysis is usually based on two approaches \cite{TimeSeriesAnalysisCryerChan}: in the (natural) time and transformed domains (for instance, frequency and wavelet).
In the context of time-domain analysis, a new methodology was proposed by \Mycite{PermutationEntropyBandtPompe}.
Its approach is non-parametric and based on descriptors of Information Theory.
Through this, the time series is transformed into ordinal patterns, with which a histogram is formed.
Using such patterns, the resulting distribution becomes less sensitive to outliers and, as it does not depend on any model, can be applied to a variety of situations.

The Bandt-Pompe methodology and its variants have been used successfully in the analysis of many types of dynamics, receiving so far more than \num{2500} citations, according to the Journal of Citation Records.
We found works using such an approach in multiple areas of scientific knowledge such as, for example,
the study of electroencephalography signals using wavelet decomposition~\cite{EEGAnalysisWaveletInformationTools},
characterization of household appliances through their energy consumption~\cite{CharacterizationElectricLoadInformationTheoryQuantifiers},
 online signature classification and verification~\cite{ClassificationVerificationOnlineHandwrittenSignatures}.

Each time series is described by a point in the range of $\mathbbm R^2$, the entropy complexity plane.
Two points are well known in this plane: those of white noise and a completely deterministic sequence.
Through these references, we can characterize the time series according to the dynamics of its generating process.
Based on this premise, studies with different applications managed to obtain relevant results from time series through information on the nature of the data provided by the $H \times C$ plane.
Examples include the analysis of \Mycite{echegoyen2020permutation} in magnetoencephalography recordings of individuals suffering from mild cognitive impairment and individuals diagnosed with Alzheimer's disease by trajectories in the $H \times C$ plane,
\Mycite{InformationTheoryPerspectiveNetworkRobustness} verified the effect of attacks on complex networks by displacing their points in the $H \times C$ plane,
\Mycite{CharacterizationVehicleBehaviorInformationTheory} described the behavior of vehicles depending on the topology of cities, and
\Mycite{Chagas2020Characterization} succeeded in expanding the use of such techniques for analyzing SAR texture images, making characterization, and classification of them.


%and \citeauthor{StructuralChangesDataCommunicationWSN}~\ycite{StructuralChangesDataCommunicationWSN} employ Information-Theoretic measures to describe the evolution of wireless sensors networks.

In the context of theoretical work, chaotic and random component analysis has been developed with the aim of improving the understanding of the plane's properties. We can cite as highlights:
\Mycite{GeneralizedStatisticalComplexityMeasuresGeometricalAnalyticalProperties} analyzed the logistic chaotic map and discuss the boundaries of the $H \times C$ plane.~\citeyear{De_Micco_2009} studied chaotic components in pseudo-random number generators.
\Mycite{DistinguishingNoiseFromChaos}  tackled the often hard problem of distinguishing chaos and noise.
\Mycite{DistinguishingChaoticStochasticDynamicsTimeSeriesMultiscaleSymbolicApproach} used a multi-scale approach to analyze the interplay between chaotic and stochastic dynamics.
With the knowledge of the expected variability of such points, according to the underlying dynamics, we can make hypothesis tests for a wide variety of models.
Some preliminary results can already be found in the literature.
\Mycite{RandomNumberGeneratorsCausality} showed that the Entropy-Complexity plane ($H\times C$) is a good indicator of the results of Diehard tests for pseudorandom number generators.
\Mycite{De_Micco_2008} assessed ways of improving pseudorandom sequences by their representation in this plane.
\Mycite{LiborInvisibleHand} identified spurious interventions in the Libor market using the $H\times C$ plane representation.

Motivated by such works, in this paper we advance the state-of-the-art providing confidence regions for white noise points in the $H\times C$ plane.
In the proposed approach, the input is a sequence of true random observations generated by a physical procedure, and the confidence regions are obtained by performing an orthogonal projection of the data onto a space of principal components analysis (PCA), thus eliminating the restrictions imposed by the curvilinear space of the Entropy-Complexity Plane.
Our contributions can be summarized as follows:
\begin{itemize}
    \item Provide the first contribution in construction that confidence regions in Entropy-Complexity Plane.
    \item Evaluate the discrimination power of these regions in the analysis of small random sequences generated by physical procedures and pseudorandom generators (PRNGs).
\end{itemize}

The paper is structured as follows: Section~\ref{Sec:Intro} introduces the elements of the study (the Bandt and Pompe methodology, the random deviates, and model).
The confidence regions are presented in Section~\ref{Sec:Results} with some analysis of this application, and the conclusions are discussed in Section~\ref{Sec:Conclusions}.

\section{Entropy-Complexity plane in the literature}

The first works on the characterization of white noises with permutation entropy arose from the need to discriminate them in relation to chaotic maps~\cites{rosso2013characterization, xiong2020complexity, olivares2012contrasting}.
However, it was found that the measure of statistical complexity was able to efficiently quantify the performance of pseudorandom number generators, expanding the possibilities of using information theory descriptors with the Bandt-Pompe symbolization~\cites{larrondo2002statistical, gonzalez2005statistical}.

Table~\ref{Tab:Literature} presents a summary of the main works in the literature that perform analysis of non-chaotic algorithmic generators, according to their features $(h, c)$.
For this, we also provide the length $T$ and embedding dimension $D$ applied in the analysis of the time series.
The following algorithmic generators were analyzed:
\begin{itemize}
    %\item[$-$] RNG available in Intel fortran compiler (FOR)
    %\item[$-$] RNG available in Borland C++ compiler (CCC)
    %\item[$-$] Matlab RAND function (MAT)
    \item Mother RNG, available in Marsaglia website~\cite{marsaglia1994yet} (MOT);
    \item Multiple with carry RNG (MWC)~\cite{marsaglia1994yet};
    \item Combo RNG (COM)~\cite{marsaglia1994yet};
    \item Lehmer RNG (LEH)~\cite{payne1969coding};
    \item Fractional Gaussian noise with $\alpha = 0$ (fGn);
    \item Fractional Brownian motion with $\alpha = 1.2$ (fBm);
    \item $f^{-k}$ noise with $k = 0$;
    \item Linear Congruential Generator (LCG)~\cite{knuth1997sorting}.
\end{itemize}

\begin{table}[H]
    \caption{Result of the main works of white noise sequences analysis in the $H \times C$ plane.}
    \label{Tab:Literature}
    \centering
    \begin{tabular}{llcccccc}
    \toprule
Reference & PRNG & $T$ & $D$ & $H$ & $C$ & Is white noise? & $p$-value\\ 
\midrule
\citeauthor{larrondo2002statistical} (\citeyear{larrondo2002statistical}) &  MOT & NA & 6 & $\cong 0.9969$ & $\cong 0$ & no & NA\\
%&  FOR & NA & 6 & $\cong 0.997$ & $\cong 0$ & no & NA\\
%&  CCC & NA & 6 & $\cong 0.997$ & $\cong 0$ & no & NA\\
%&  MAT & NA & 6 & $\cong 0.997$ & $\cong 0$ & no & NA\\
\cmidrule(lr){1-8}
\citeauthor{gonzalez2005statistical} (\citeyear{gonzalez2005statistical})  &  MWC & 65536 & NA & $\cong 1$ & $0.3$ & yes & NA\\
 &  MOT & 65536 & NA & $\cong 1$ & $0.3$ & yes & NA\\
 &  COM & 65536 & NA & $\cong 1$ & $0.05$ & yes & NA\\
\cmidrule(lr){1-8}
\citeauthor{RandomNumberGeneratorsCausality} (\citeyear{RandomNumberGeneratorsCausality}) &  LEH & \num[scientific-notation=true]{5 e6} & 5 & NA & $10^{-4}$ & yes & NA\\
 &  MOT & \num[scientific-notation=true]{5 e6} & 5 & NA & $10^{-4}$ & yes & NA\\
 &  MWC & \num[scientific-notation=true]{5 e6} & 5 & NA & $10^{-4}$ & yes & NA\\
\cmidrule(lr){1-8}
\citeauthor{olivares2012contrasting} (\citeyear{olivares2012contrasting}) &  fGn & \num[scientific-notation=true]{2 e15} & 6 & $\cong 0.998$ & NA & yes & NA\\
 & fBm & \num[scientific-notation=true]{2 e15} & 6 & $\cong 0.993$ & NA & yes & NA\\
 & $f^{-k}$ & \num[scientific-notation=true]{2 e15} & 6 & $\cong 0.997$ & NA & yes & NA\\
\cmidrule(lr){1-8}
\citeauthor{rosso2013characterization} (\citeyear{rosso2013characterization}) &  LCG & \num[scientific-notation=true]{1 e7} & 6 & $0.997871$ & $0.005101$ & no & NA\\
\cmidrule(lr){1-8}
\citeauthor{xiong2020complexity} (\citeyear{xiong2020complexity}) &  fGn & \num[scientific-notation=true]{2 e17} & 6 & $\cong 1$ & $\cong 0$ & yes & NA\\
 & $f^{-k}$ & \num[scientific-notation=true]{2 e17} & 6 & $\cong 1$ & $\cong 0$ & yes & NA\\
\bottomrule
    \end{tabular}
\end{table}


\section{Bandt and Pompe Symbolization: A Background}\label{Sec:BP}

In our work, we consider the ordinal patterns formed from true white noise sequences (TWNS) using the Bandt and Pompe symbolization.
Such sequences are then mapped in the two-dimensional plane of Information Theory descriptors, formed by the Permutation Entropy and the Statistical Complexity.
We then obtain a test statistic, using confidence regions, that can discriminate among times series.

\subsection{The Bandt and Pompe Methodology}\label{Sec:BPMethodology}

Let ${\mathcal X} \equiv \{x_t\}_{t=1}^{T}$ be a real-valued time series of length $T$, without ties. 
As stated by \Mycite{PermutationEntropyBandtPompe} in their seminal work:  
\begin{quote}
	``If the $\{x_t\}_{t=1}^{T}$ attain infinitely many values, it is common to replace them by a symbol sequence 
	$\Pi \equiv \{\pi_j\}$ with finitely many symbols, and calculate source entropy from it".
\end{quote}
Also, as stressed by these authors, 
\begin{quote}
	``The corresponding symbol sequence must come 
	naturally from the $\{x_t\}_{t=1}^{T}$ without former model assumptions".
\end{quote}

Let ${\mathbbm A}_{D}$ (with $D \geq 2$ and $D \in {\mathbbm N}$) be the symmetric group of order $D!$ formed by all 
possible permutation of order $D$, and the symbol component vector 
${\bm \pi}^{(D)} = (\pi_1, \pi_2, \dots, \pi_D)$ so every element ${\bm \pi}^{(D)}$ is unique 
($\pi_j \neq \pi_k~\forall~j \neq k$). 
Consider for the time series ${\mathcal X} \equiv \{x_t\}_{t=1}^{T}$ its time delay embedding representation,
with embedding dimension $D \geq 2$ ($D \in {\mathbbm N}$) and time delay $\tau \geq 1$ ($\tau \in {\mathbbm N}$, also called ``embedding time''):
\begin{equation} 
	{\mathbf X}^{(D,\tau)}_t =( x_t,x_{t+\tau},\dots,x_{t+(D-1)\tau} ) ,
	\label{eq:time-delay}
\end{equation} 
for $t = 1,2,\dots,N$ with $N = T-(D-1) \tau$.
Then, the vector ${\mathbf X}^{(D,\tau)}_t$ can be mapped to a symbol ${\bm \pi}^{(D)} \in {\mathbbm A}_{D}$. 
This mapping should be defined in a way that preserves the desired relation between the elements 
$x_t  \in {\mathbf X}^{(D,\tau)}_t$, and all $t \in T$ that share this pattern (also called ``motif'') are mapped to the same 
${\bm \pi}^{(D)}$. 
The two most frequent ways to define the mapping ${\mathbf X}^{(D,\tau)} \mapsto {\bm \pi}^{(D)}$ are:  
\begin{enumerate}[label=\alph*)]
	\item ordering the ranks of $x_t \in {\mathbf X}^{(D,\tau)}$ in chronological order 
	(\textit{Rank Permutation}) or,
	\item ordering the time indexes of $x_t \in {\mathbf X}^{(D,\tau)}$  
	(\textit{Chronological Index Permutation}).
\end{enumerate}
See details in the work by \citet{BPRepeatedValuesChaos}.
Without loss of generality, in the following, we will use the latter.

Consider, for instance, the time series $\mathcal X = (2.8, 2.2, 4.2, 5.8, 5.2, 5.5, 3.3, 4.7, 2.2, 1.5)$ depicted in Fig.~\ref{Fig:IntroBP} as a light blue line.
Assume we are using patterns of length $D=5$ with a unitary time lag $\tau=1$.
The code associated to $\mathbf X_{3}^{(5,1)}=(x_3,\dots,x_7)=(4.2, 5.8, 5.2, 5.5, 3.3)$, shown in red, is formed by the indexes in $\bm\pi^{(5)}=(1,2,3,4,5)$ which sort the elements of $\mathbf X_{3}^{(5,1)}$ in increasing order: $51342$.
With this, $\widetilde{\pi}^{(5)} = 51342$, and we increase the counting related to this motif in the histogram of all possible patterns of size $D=5$.

The green line in Fig.~\ref{Fig:IntroBP} illustrates $\mathbf X_{1}^{(5,2)}$, i.e. the sequence of length $D=5$ starting at $x_1$ with lag $\tau=2$.
In this case, $\mathbf X_{1}^{(5,2)}= (2.8, 4.2, 5.2, 3.3, 2.2)$, and the corresponding motif is $\widetilde{\pi}^{(5)}=51423$.

\begin{figure}[hbt]
	\centering
	\includegraphics[width=.7\linewidth]{IntroBP}
	\caption{Illustration of the Bandt and Pompe coding}
	\label{Fig:IntroBP}
\end{figure}

After computing all the symbols, one obtains the histogram of proportions $\bm h = (h(j))_{1\leq j\leq D!}$.
Such histogram estimates the (unknown, in general) probability distribution function of these patterns.
The next step into the characterization of the time series is computing descriptors from this histogram.

The first descriptor is a measure of the disorder of the system.
The most frequently used feature for this is the Normalized Shannon entropy, defined as
\begin{equation}
	H(\bm h) = -\frac{1}{\log D!} \sum_{j=1}^{D!} h(j) \log h(j),
\end{equation}
with the convention that terms in the summation for which $h(j)=0$ are null.
This quantity is bounded in the unit interval. 
It is zero when $h(j)=1$ for some $j$ (and, thus, all other bins are zero), and one when $h(j)=1/D!$ for every $j$ (the uniform probability distribution function).

Although very expressive, the Normalized Shannon Entropy is not able to describe all possible underlying dynamics.
In particular, for intermediate values of $H$, there is a wide variety of situations worth characterizing.
To this aim, \citet{LopezRuiz1995} proposed using the disequilibrium  $Q$, a measure of how far $\bm h$ is from an equilibrium or noninformative distribution.
They employed the Euclidean distance between $\bm h$ and the uniform probability distribution function.

With this, they proposed $C=HQ$ as a measure of the Statistical Complexity of the underlying dynamics.
A time series can then be mapped into a point in the $H\times C$ plane.

\subsection{The Entropy-Complexity Plane}\label{Sec:HCPlane}

The Entropy-Complexity plane is the set of all possible points $(h,c)$ that can be produced by arbitrary time series analyzed with embedding dimension $D$ that are mapped on histograms of $D!$ bins.
The time delay is irrelevant, and we consider infinitely long series.

Let us consider two extreme cases:
\begin{enumerate}[label=Case~\Roman*., align=left, leftmargin=*]
	\item 	Strictly monotonically increasing or decreasing series produce a single pattern, so the other $D!-1$ bins of the histogram are zero. 
	The entropy is zero, and the distance to the uniform distribution is maximal. 
	Therefore, the complexity is zero, and such series are mapped onto the point $(0,0)$.
	\item 	White noise produces a histogram of equal proportions $1/D!$ and maximal entropy. 
	The distance to the equilibrium distribution is zero. 
	Thus, such series are mapped onto the point $(1,0)$.
\end{enumerate}

\citet{SomeFeaturesoftheLMCStatisticalComplexity} proved that, for a fixed value of entropy, there are two extreme values of complexity.
\citet{martin2006generalized}, using geometrical arguments on the space of configurations, found expressions for such boundaries.
The lower boundary $C_{\min}$ is smooth, while the upper $C_{\max}$ is defined by $D!-1$ pieces.
The upper boundary converges to a smooth curve when $D\to\infty$.


Fig.~\ref{fig:Boundaries} shows the boundaries of the $H\times C$ plane for the embedding dimensions $D=3$ (red) $D=4$ (green), and $D=5$ (blue).
The inset plot highlights the fine structure of the upper boundary inside the rectangle.
The jagged structure of $C_{\max}$ increases the difficulty of finding distributions for the points in the $H\times C$ plane.

\begin{figure}[hbt]
	\centering
	\includegraphics[width=.7\linewidth]{Figures/BoundariesPlot}
	\caption{Boundaries of the $H\times C$ plane for dimension embeddings $D=3,4,5$.}\label{fig:Boundaries}
\end{figure}

We illustrate the use of the Entropy-Complexity plane ($H\times C$) with the following time series:
\begin{itemize}
	\item Colored $k$-noise, or $f^{-k}$ noise: white ($k=0$), $k=1/2$, pink ($k=1$), $k=3/2$, red ($k=2$), $k=5/2$, and $k=3$;
	\item Chaotic logistic series $x_t = r x_{t-1} (1 - x_{t-1})$, with $r=3.6$ and $4$;
	\item Deterministic series: monotonic increasing ($\log(x_t+0.1)$, $x_t=\{1,2,\dots,10^4$) and periodic ($\sin(2x_t)\cos(2x_t)$, with $0\leq x_t\leq 2\pi$ over ten thousand equally spaced points).
\end{itemize}
In all cases, we used $D=6$ and $\tau=1$.
Fig.~\ref{fig:Histograms} shows nine of the histograms produced by these series using the Mersenne-Twister pseudorandom number generator;
we omitted those corresponding to the deterministic series, as they produce one and two nonzero bins.

\begin{figure}[hbt]
	\includegraphics[width=\linewidth]{Figures/h.pdf}
	\caption{Patterns histograms of selected time series  for dimension embedding $D = 6$, time delay $\tau = 1$, and sequence length $T = \num[scientific-notation=true]{e4}$.}
	\label{fig:Histograms}
\end{figure}

Fig.~\ref{fig:AllSystems} shows the $H\times C$ plane with the bounds for $D=6$, the time series, and the points they were mapped onto.
The points due to $f^{-k}$ noises appear joined by dotted segments.
It is noticeable that deterministic patterns have more complexity than random ones.
Also, points related to $f^{-k}$ noises tend to clutter for $k<1$, having the highest entropy values, as can be seen in Fig.~\ref{fig:RightMostCorner}.

\begin{figure}[hbt]
	\centering
	\includegraphics[width=\linewidth]{AllSystems}
	\caption{Eleven systems and their points in the $H\times C$ plane for dimension embedding $D = 6$, time delay $\tau = 1$, and sequence length $T = \num[scientific-notation=true]{e4}$.}
	\label{fig:AllSystems}
\end{figure}

Fig.~\ref{fig:RightMostCorner} shows the rightmost lower corner of the $H\times C$ plane, emphasizing the location of the white ($k=0$), $k=1/2$, and pink ($k=1$) noises.

\begin{figure}[hbt]
	\centering
	\includegraphics[width=\linewidth]{RightMostCorner}
	\caption{Representations of white noise, $f^{-1/2}$, and $f^{-1}$ noise in the $H \times C$ plane  for dimension embedding $D = 6$, time delay $\tau = 1$, and sequence length $T = \num[scientific-notation=true]{e4}$.}
	\label{fig:RightMostCorner}
\end{figure}

Due to the infinitude of white noise sequences, although these sequences have the characteristic of presenting high values of entropy and low statistical complexity, their points will not necessarily be located in $(1, 0)$, but in a surrounding region.
Our study's focus is to assess pure randomness by analyzing the empirical distribution of the points produced by true random sequences of finite size and obtaining regions of confidence in the $H \times C$ plane.

\section{Entropy-Complexity plane in the literature}

The first works on the characterization of white noises with permutation entropy arose from the need to discriminate them in relation to chaotic maps~\cites{rosso2013characterization, xiong2020complexity, olivares2012contrasting}.
However, it was found that the measure of statistical complexity was able to efficiently quantify the performance of pseudorandom number generators, expanding the possibilities of using information theory descriptors with the Bandt-Pompe symbolization~\cites{larrondo2002statistical, gonzalez2005statistical}.

Table~\ref{Tab:Literature} presents a summary of the main works in the literature that perform analysis of non-chaotic algorithmic generators, according to their features $(h, c)$.
For this, we also provide the length $T$ and embedding dimension $D$ applied in the analysis of the time series.
The following algorithmic generators were analyzed:
\begin{itemize}
    %\item[$-$] RNG available in Intel fortran compiler (FOR)
    %\item[$-$] RNG available in Borland C++ compiler (CCC)
    %\item[$-$] Matlab RAND function (MAT)
    \item Mother RNG, available in Marsaglia website~\cite{marsaglia1994yet} (MOT);
    \item Multiple with carry RNG (MWC)~\cite{marsaglia1994yet};
    \item Combo RNG (COM)~\cite{marsaglia1994yet};
    \item Lehmer RNG (LEH)~\cite{payne1969coding};
    \item Fractional Gaussian noise with $\alpha = 0$ (fGn);
    \item Fractional Brownian motion with $\alpha = 1.2$ (fBm);
    \item $f^{-k}$ noise with $k = 0$;
    \item Linear Congruential Generator (LCG)~\cite{knuth1997sorting}.
\end{itemize}

\begin{table}[H]
    \caption{Result of the main works of white noise sequences analysis in the $H \times C$ plane.}
    \label{Tab:Literature}
    \centering
    \begin{tabular}{llcccccc}
    \toprule
Reference & PRNG & $T$ & $D$ & $H$ & $C$ & Is white noise? & $p$-value\\ 
\midrule
\citeauthor{larrondo2002statistical} (\citeyear{larrondo2002statistical}) &  MOT & NA & 6 & $\cong 0.9969$ & $\cong 0$ & no & NA\\
%&  FOR & NA & 6 & $\cong 0.997$ & $\cong 0$ & no & NA\\
%&  CCC & NA & 6 & $\cong 0.997$ & $\cong 0$ & no & NA\\
%&  MAT & NA & 6 & $\cong 0.997$ & $\cong 0$ & no & NA\\
\cmidrule(lr){1-8}
\citeauthor{gonzalez2005statistical} (\citeyear{gonzalez2005statistical})  &  MWC & 65536 & NA & $\cong 1$ & $0.3$ & yes & NA\\
 &  MOT & 65536 & NA & $\cong 1$ & $0.3$ & yes & NA\\
 &  COM & 65536 & NA & $\cong 1$ & $0.05$ & yes & NA\\
\cmidrule(lr){1-8}
\citeauthor{RandomNumberGeneratorsCausality} (\citeyear{RandomNumberGeneratorsCausality}) &  LEH & \num[scientific-notation=true]{5 e6} & 5 & NA & $10^{-4}$ & yes & NA\\
 &  MOT & \num[scientific-notation=true]{5 e6} & 5 & NA & $10^{-4}$ & yes & NA\\
 &  MWC & \num[scientific-notation=true]{5 e6} & 5 & NA & $10^{-4}$ & yes & NA\\
\cmidrule(lr){1-8}
\citeauthor{olivares2012contrasting} (\citeyear{olivares2012contrasting}) &  fGn & \num[scientific-notation=true]{2 e15} & 6 & $\cong 0.998$ & NA & yes & NA\\
 & fBm & \num[scientific-notation=true]{2 e15} & 6 & $\cong 0.993$ & NA & yes & NA\\
 & $f^{-k}$ & \num[scientific-notation=true]{2 e15} & 6 & $\cong 0.997$ & NA & yes & NA\\
\cmidrule(lr){1-8}
\citeauthor{rosso2013characterization} (\citeyear{rosso2013characterization}) &  LCG & \num[scientific-notation=true]{1 e7} & 6 & $0.997871$ & $0.005101$ & no & NA\\
\cmidrule(lr){1-8}
\citeauthor{xiong2020complexity} (\citeyear{xiong2020complexity}) &  fGn & \num[scientific-notation=true]{2 e17} & 6 & $\cong 1$ & $\cong 0$ & yes & NA\\
 & $f^{-k}$ & \num[scientific-notation=true]{2 e17} & 6 & $\cong 1$ & $\cong 0$ & yes & NA\\
\bottomrule
    \end{tabular}
\end{table}


\section{Methodology}\label{Sec:Method}

\subsection{The Bandt-Pompe Methodology}

Let ${\mathcal X} \equiv \{x_t\}_{t=1}^{T}$ be a real valued time series of length $T$, without ties. 
As stated
by~\citeauthor{Bandt2002Permutation}~\citeyear{Bandt2002Permutation}in their seminal work:  
\begin{quote}
``If the $\{x_t\}_{t=1}^{T}$ attain infinitely many values, it is common to replace them by a symbol sequence 
$\Pi \equiv \{\pi_j\}$ with finitely many symbols, and calculate source entropy from it".
\end{quote}
Also, as stressed by these authors, 
\begin{quote}
``The corresponding symbol sequence must come 
naturally from the $\{x_t\}_{t=1}^{T}$ without former model assumptions".
\end{quote}

Let ${\mathbbm A}_{D}$ (with $D \geq 2$ and $D \in {\mathbbm Z}$) be the symmetric group of order $D!$ formed by all 
possible permutation of order $D$, and the symbol component vector 
${\bm \pi}^{(D)} = (\pi_1, \pi_2, \dots, \pi_D)$ so every element ${\bm \pi}^{(D)}$ is unique 
($\pi_j \neq \pi_k~\forall~j \neq k$). 
Consider for the time series ${\mathcal X} \equiv \{x_t\}_{t=1}^{T}$ its time delay embedding representation,
with embedding dimension $D \geq 2$ ($D \in {\mathbbm Z}$) and time delay $\tau \geq 1$ ($\tau \in {\mathbbm Z}$, also called ``embedding time''):
\begin{equation} 
\label{eq:time-delay}
{\mathbf X}^{(D,\tau)}_t ~=~( x_t,x_{t+\tau},\dots,x_{t+(D-1)\tau} ) \ ,
\end{equation} 
for $t = 1,2,\dots,N$ with $N = T-(D-1) \tau$.
Then the vector ${\mathbf X}^{(D,\tau)}_t$ can be mapped to a symbol vector ${\bm \pi}^{(D)} \in {\mathbbm A}_{D}$. 
This mapping should be defined in a way that preserves the desired relation between the elements 
$x_t  \in {\mathbf X}^{(D,\tau)}_t$, and all $t \in T$ that share this pattern (also called motif) have to mapped to the same 
${\bm \pi}^{(D)}$. 
The two most frequent ways to define the mapping ${\mathbf X}^{(D,\tau)} \mapsto {\bm \pi}^{(D)}$ are:  
\begin{enumerate}[a)]
\item ordering the ranks of the $x_t \in {\mathbf X}^{(D,\tau)}$ in chronological order 
       (\textit{Rank Permutation}) or,
\item ordering the time indexes according to the ranks of $x_t \in {\mathbf X}^{(D,\tau)}$  
       (\textit{Chronological Index Permutation});
\end{enumerate}
       see details in \citeauthor{Traversaro2018Bandt}~\ycite{Traversaro2018Bandt}.
Without loss of generality, in the following we will only use the latter.

Consider, for instance, the time series $\mathcal X = (1.8, 1.2, 3.2, 4.8, 4.2, 4.5, 2.3, 3.7, 1.2, .5)$ depicted in Fig.~\ref{Fig:IntroBP}.
Assume we are using patterns of length $D=5$ with unitary time lag $\tau=1$.
The code associated to $\mathbf X_{3}^{(5,1)}=(x_3,\dots,x_7)=(3.2, 4.8, 4.2, 4.5, 2.3)$, shown in black, is formed by the indexes in $\bm\pi^{(5)}=(1,2,3,4,5)$ which sort the elements of $\mathbf X_{3}^{(5,1)}$ in increasing order: $51342$.
With this, $\widetilde{\pi}^{(5)} = 51342$, and we increase the counting related to this motif in the histogram of all possible patterns of size $D=5$.

The dash-dot line in Fig.~\ref{Fig:IntroBP} illustrates $\mathbf X_{1}^{(5,2)}$, i.e. the sequence of length $D=5$ starting at $x_1$ with lag $\tau=2$.
In this case, $\mathbf X_{1}^{(5,2)}= (1.8, 3.2, 4.2, 2.3, 1.2)$, and the corresponding motif is $\widetilde{\pi}^{(5)}=51423$.

\begin{figure}[H]
\centering
\includegraphics[width=.7\linewidth]{IntroBP}
\caption{Illustration of the Bandt and Pompe coding}
\label{Fig:IntroBP}
\end{figure}

Once all symbols have been computed, one obtains the histogram of proportions $\bm h = (h(j))_{1\leq j\leq D!}$.
This is an estimate of the (unknown, in general) probability distribution function of these patterns.
The next step into the characterization of the time series is computing descriptors from this histogram.

The first descriptor is a measure of the disorder of the system.
The most frequently used feature for this is the Normalized Shannon entropy, defined as
\begin{equation}
H(\bm h) = -\frac{1}{\log D!} \sum_{j=1}^{D!} h(j) \log h(j),
\end{equation}
with the convention that terms in the summation for which $h(j)=0$ are null.
This quantity is bounded in the unit interval, and is zero when $h(j)=1$ for some $j$ (and, thus, all other bins are zero), and one when $h(j)=1/D!$ for every $j$ (the uniform probability distribution function).

Although very expressive, the Normalized Shannon Entropy is not able to describe all possible underlying dynamics.
In particular, for intermediate values of $H$, there is a wide variety of situations worth characterizing.
To this aim, \citeauthor{LopezRuiz1995}~\citeyear{LopezRuiz1995} proposed using $Q$, the disequilibrium, a measure of how far $\bm h$ is from an equilibrium or noninformative distribution.
They employed the Euclidean distance between $\bm h$ and the uniform probability distribution function.

With this, they proposed $C=HQ$ as a measure of the Statistical Complexity of the underlying dynamics.
A time series can then be mapped into a point in the $H\times C$ plane.


\subsection{The Entropy-Complexity Plane}

We illustrate the use of the Entropy-Complexity ($H\times C$) with the following time series:
\begin{itemize}
\item Colored $k$-noise: white ($k=0$), $k=-1/2$, pink ($k=1$), $k=3/2$, red ($k=2$), $=5/2$, and $k=3$;
\item Chaotic logistic series $x_t = r x_{t-1} (1 - x_{t-1})$, with $r=3.6$ and $4$;
\item Deterministic series: monotonic increasing ($\log(x_t+0.1)$, $x_t=\{1,2,\dots,10^4$) and periodic ($\sin(2x_t)\cos(2x_t)$, with $0\leq x_t\leq 2\pi$ over ten thousand equally spaced points).
\end{itemize}
In all cases, we used $D=6$ and $\tau=1$.
Fig.~\ref{fig:Histograms} shows nine of the histograms produced by these series using the Mersenne-Twister pseudorandom number generator;
we omitted those corresponding to the deterministic series, as they produce one and two nonzero bins.

\begin{figure}[H]
\centering
	\subfloat[Logistic map $r=3.6$]{\includegraphics[width=.3\linewidth]{h36}}\quad
	\subfloat[Logistic map $r=4$]{\includegraphics[width=.3\linewidth]{h4}}\quad
	\subfloat[$f^{-3}$ noise]{\includegraphics[width=.3\linewidth]{h3}}\quad
	\subfloat[$f^{-5/2}$ noise]{\includegraphics[width=.3\linewidth]{h25}}\quad
	\subfloat[$f^{-2}$ noise]{\includegraphics[width=.3\linewidth]{h2}}\quad
	\subfloat[$f^{-1.5}$ noise]{\includegraphics[width=.3\linewidth]{h15}}\quad
	\subfloat[$f^{-1}$ noise]{\includegraphics[width=.3\linewidth]{h1}}\quad
	\subfloat[$f^{-1/2}$ noise]{\includegraphics[width=.3\linewidth]{h05}}\quad
	\subfloat[White noise]{\includegraphics[width=.3\linewidth]{h0}}
	\caption{Patterns histograms of selected time series, with $D=6$ and $\tau=1$\label{fig:Histograms}}
\end{figure}


Fig.~\ref{fig:AllSystems} shows the $H\times C$ plane with the bounds for $D=6$, the time series and the points they were mapped onto.
The points due to $f^{-k}$ noises appear joined by dotted segments.
It is noticeable that deterministic patterns have more complexity than random ones.
Also, points related to $f^{-k}$ noises tend to clutter for $k<1$.

\begin{figure}[H]
\centering
\includegraphics[width=\linewidth]{AllSystems}
\caption{Eleven systems and their points in the $H\times C$ plane}\label{fig:AllSystems}
\end{figure}

Fig.~\ref{fig:RightMostCorner} shows the rightmost lower corner of the $H\times C$ plane, emphasizing the location of the white ($k=0$), $k=-1/2$, and pink ($k=1$) noises.

\begin{figure}[H]
\centering
\includegraphics[width=\linewidth]{RightMostCorner}
\caption{White Noise, $f^{-1/2}$ and $f^{-1}$ noise points}\label{fig:RightMostCorner}
\end{figure}

The focus of our study is the empirical distribution of points produced by White and selected $f^{-k}$ noises, with this, providing confidence regions in the $H\times C$ plane.
In particular, we are interested in assessing pure randomness with this technique.

\subsection{The $H\times C$ plane in the literature}

Table~\ref{Tab:Literature} presents references in which the authors analyze time series of size $N$ and embedding dimension $D$.
The authors also attest whether the time series is white noise or not, according to its $(h,c)$ feature.

The following widely used non-chaotic algorithmic generators are analyzed:
\begin{itemize}
    \item[$-$] RNG available in Intel fortran compiler (FOR)
    \item[$-$] RNG available in Borland C++ compiler (CCC)
    \item[$-$] Matlab RAND function (MAT)
    \item[$-$] Mother RNG, available in Marsaglia website~\cite{marsaglia1994yet} (MOT)
    \item[$-$] Multiple with carry RNG (MWC)~\cite{marsaglia1994yet}
    \item[$-$] Combo RNG (COM)~\cite{marsaglia1994yet}
    \item[$-$] Lehmer RNG (LEH)~\cite{payne1969coding}
    \item[$-$] Fractional Gaussian noise with $\alpha = 0$ (fGn)
    \item[$-$] Fractional Brownian motion with $\alpha = 1.2$ (fBm)
    \item[$-$] $f^{-k}$ noise with $k = 0$
    \item[$-$] Linear Congruential Generator (LCG)~\cite{knuth1997sorting}
\end{itemize}

\begin{table}[H]
    \caption{References which attest whether a time series of size $N$ is white noise according to its $(h,c)$ feature, and the empirical $p$-value of such hypothesis according to our results}
    \label{Tab:Literature}
    \centering
    \begin{tabular}{llcccccc}
    \toprule
Reference & PRNG & $N$ & $D$ & $H$ & $C$ & Is white noise? & $p$-value\\ 
\midrule
\citeauthor{larrondo2002statistical} (\citeyear{larrondo2002statistical}) &  MOT & NA & 6 & $\cong 0.9969$ & $\cong 0$ & no & NA\\
 &  FOR & NA & 6 & $\cong 0.997$ & $\cong 0$ & no & NA\\
 &  CCC & NA & 6 & $\cong 0.997$ & $\cong 0$ & no & NA\\
 &  MAT & NA & 6 & $\cong 0.997$ & $\cong 0$ & no & NA\\
\cmidrule(lr){1-8}
\citeauthor{gonzalez2005statistical} (\citeyear{gonzalez2005statistical})  &  MWC & 65536 & NA & $\cong 1$ & $0.3$ & yes & NA\\
 &  MOT & 65536 & NA & $\cong 1$ & $0.3$ & yes & NA\\
 &  COM & 65536 & NA & $\cong 1$ & $0.05$ & yes & NA\\
\cmidrule(lr){1-8}
\citeauthor{Larrondo2006Random} (\citeyear{Larrondo2006Random}) &  LEH & \num[scientific-notation=true]{5 e6} & 5 & NA & $10^{-4}$ & yes & NA\\
 &  MOT & \num[scientific-notation=true]{5 e6} & 5 & NA & $10^{-4}$ & yes & NA\\
 &  MWC & \num[scientific-notation=true]{5 e6} & 5 & NA & $10^{-4}$ & yes & NA\\
\cmidrule(lr){1-8}
\citeauthor{olivares2012contrasting} (\citeyear{olivares2012contrasting}) &  fGn & \num[scientific-notation=true]{2 e15} & 6 & $\cong 0.998$ & NA & yes & NA\\
 & fBm & \num[scientific-notation=true]{2 e15} & 6 & $\cong 0.993$ & NA & yes & NA\\
 & $f^{-k}$ & \num[scientific-notation=true]{2 e15} & 6 & $\cong 0.997$ & NA & yes & NA\\
\cmidrule(lr){1-8}
\citeauthor{rosso2013characterization} (\citeyear{rosso2013characterization}) &  LCG & \num[scientific-notation=true]{1 e7} & 6 & $0.997871$ & $0.005101$ & no & NA\\
\cmidrule(lr){1-8}
\citeauthor{xiong2020complexity} (\citeyear{xiong2020complexity}) &  fGn & \num[scientific-notation=true]{2 e17} & 6 & $\cong 1$ & $\cong 0$ & yes & NA\\
 & $f^{-k}$ & \num[scientific-notation=true]{2 e17} & 6 & $\cong 1$ & $\cong 0$ & yes & NA\\
\bottomrule
    \end{tabular}
\end{table}

%The mathematical theory of markets states that fairness is characterized by white noise.

\subsection{True Random Numbers}

\textcolor{red}{Gerador - Marcelo e Heitor}

Random numbers are used in many fields, from gambling to criptography, aiming to guarantee a secure, realistic or unpredictable behavihor. 
Pseudo randomic results can be achieved by software in a deterministic way. 
But, some applications need actual random numbers (despite the somewhat elusive nature of actual randomness).
Randomness can be observed in unpredictable real world phenomena like cathodic radiation or atmospheric noise.
In this study we used two sources of real random numbers. 
The first is based on vacuum states to generate random quantum numbers described by \citeauthor{Wittmann2010generator}~(\citeyear{Wittmann2010generator}), the second one is based on atmospheric noise captured by a cheap radio receiver presented at \url{www.random.org}.

%Deixei as URLs pois não tenho o .bib ...

%The colored $k$-noise mentioned in the previous section, also known as ``correlated noise'', is such that its power spectrum has shape $f^{-k}$.
%When $k=0$, we have white noise, i.e., independent deviates.
%The larger $k$ is, the more correlated the observations are.

\subsection{Empirical Confidence Regions}

As we do not know the joint probability distribution of the pair $(h, c)$ for a sequence of random variables collectively independent and identically distributed according to a uniform law, studies involving classical bi-variate analysis, linear regression, and generalized linear models become unfeasible at this first moment.
Therefore, for the construction of our proposal, we adopted a non-parametric approach, making an empirical analysis of data obtained from physical sources and using them as our reference in the search for confidence regions.

The set of all feasible pairs in the $H \times C$ plane is found in a compact subset of $\mathbbm{R}^2$, which has limits with explicit expressions for the boundaries of this closed manifold, dependent only on the dimension of the probability space considered, that is $D!$ in the traditional Bandt-Pompe method~\cite{martin2006generalized}.
Due to such quotas, some limitations are generated, such as the absence of a representative distance metric and the difficulty of proposing confidence regions.
In view of this, it is necessary to apply an orthogonal projection in the data for a new two-dimensional coordinate system to solve these restrictions.
A classic proposal in these categories of problems is the principal component analysis (PCA) algorithm~\cite{wold1987principal}.

Let $ {\mathcal P} = \{p_n \}_{n = 1}^{N} $ be a set of $N$ observations in the $H \times C$ plane, where ${p_n} = (h_n, c_n)$ corresponds to a time series.
We apply a principal component analysis to these points and obtain $\{(u_n,v_n)\}_{n=1}^N$.
This transformation yields uncorrelated observations in the $\Omega_1\times\Omega_2$ plane.

We will find the empirical confidence regions on the $\Omega_1\times\Omega_2$ plane, and then they will be mapped back to the $H \times C$ plane.
For simplicity, and without loss of generality, assume $N$ odd; we find a parallelepiped that contains 
\SI{100}{\minusalphapercent} of the points in the $\Omega_1\times\Omega_2$ as
\begin{enumerate}
    \item Find the ranks that sort the values of the first principal component $\bm u=(u_1,u_2,\dots,u_N)$ in ascending order: $\bm r=(r_1,r_2,\dots,r_N)$, i.e., $u_{r_1}$ is the minimum value, and $u_{r_N}$ is the maximum value.
    \item Find point $(u,v)$ whose first principal component is the median: $(u_{r_{(N+1)/2}}, \cdot)$.
    \item Find the point $(u,v)$ whose first principal component is the quantile $\alpha/2$: $(u_{r_{[N\alpha/2]}}, \cdot)$.
    \item Find the point $(u,v)$ whose first principal component is the quantile $1-\alpha/2$: $(u_{r_{[N(1-\alpha/2)]}}, \cdot)$.
    \item The values $u_{r_{[N\alpha/2]}}$ and $u_{r_{[N(1-\alpha/2)]}}$ are the rightmost and leftmost bounds of the box, respectively.
    \item The top and bottom bounds of the box are the minimum and maximum values of the second principal component of the points whose first principal component is at least $u_{r_{[N\alpha/2]}}$ and at most $u_{r_{[N(1-\alpha/2)]}}$; denote these values $v_{\min}$ and $v_{\max}$, respectively.
    %\item The lower-left and upper-right corners of the box are $(u_{r_{[N\alpha/2]}}, v_{\min})$, and $(u_{r_{[N\alpha/2]}},v_{\max})$, respectively.
    \item The corners of the box are 
    $(u_{r_{[N\alpha/2]}}, v_{\min})$, 
    $(u_{r_{[N\alpha/2]}}, v_{\max})$, 
    $(u_{r_{[N(1-\alpha/2)]}}, v_{\min})$ and 
    $(u_{r_{[N(1-\alpha/2)]}},v_{\max})$.
    %\item Apply the inverse PCA transform to these corners obtaining $(h_1^{\SI{100}{\minusalphapercent}}, c_1^{\SI{100}{\minusalphapercent}})$, $(h_2^{\SI{100}{\minusalphapercent}}, c_2^{\SI{100}{\minusalphapercent}})$, $(h_1^{\SI{100}{\alphapercent}}, c_1^{\SI{100}{\alphapercent}})$ and $(h_2^{\SI{100}{\alphapercent}},c_2^{\SI{100}{\alphapercent}})$.
    \item Apply the inverse PCA transform to these corners obtaining $(h_{v_1}, c_{v_1})$, $(h_{v_2}, h_{v_2})$, $(h_{v_3}, c_{v_3})$ and $(h_{v_4},c_{v_4})$.
\end{enumerate}
These steps are also depicted in Fig.~\ref{fig:methodology}.

\begin{figure}[H]
    \centering
    \includegraphics[width=\linewidth]{Figures/methodology.pdf}
    \caption{Outline of the methodology used for the construction of confidence regions.}
    \label{fig:methodology}
\end{figure}

\begin{sidewaystable}
    \caption{Empirical confidence regions for time series of length $T = \num[scientific-notation=true]{5 e4}$ in the $H \times C$ plane obtained by the proposed methodology}
    \label{Tab:Regions50k}
    \centering
    \begin{tabular}{cccccc}
    \toprule
Confidence Regions & $D$ & $(h_{v_1}, c_{v_1})$ & $(h_{v_2}, h_{v_2})$ & $(h_{v_3}, c_{v_3})$ & $(h_{v_4},c_{v_4})$\\ 
\midrule
\SI{90}{\percent} & 3 & $(0.9999489, \num[scientific-notation=true]{5.06e-05})$ & $(0.9999487, \num[scientific-notation=true]{5.04e-05})$ & $(0.9999998, \num[scientific-notation=true]{4e-07})$ & $(0.9999996, \num[scientific-notation=true]{2e-07})$\\
\SI{95}{\percent} & 3 & $(0.9999384, \num[scientific-notation=true]{6.11e-05})$ & $(0.9999382, \num[scientific-notation=true]{6.09e-05})$ & $(0.9999994, \num[scientific-notation=true]{9e-07})$ & $(0.9999991, \num[scientific-notation=true]{7e-07})$\\
\SI{99}{\percent} & 3 & $(0.9999079, \num[scientific-notation=true]{9.11e-05})$ & $(0.9999077, \num[scientific-notation=true]{9.09e-05})$ & $(0.9999982, \num[scientific-notation=true]{2e-06})$ & $(0.999998, \num[scientific-notation=true]{1.8e-06})$\\
\SI{99.9}{\percent} & 3 & $(0.9998625, 0.0001361)$ & $(0.9998622, 0.0001358)$ & $(0.9999973, \num[scientific-notation=true]{3e-06})$ & $(0.999997, \num[scientific-notation=true]{2.7e-06})$\\
\cmidrule(lr){1-6}
\SI{90}{\percent} & 4 & $(0.9999684, \num[scientific-notation=true]{3.98e-05})$ & $(0.9999696, \num[scientific-notation=true]{4.13e-05})$ & $(0.9998075, 0.0002508)$ & $(0.9998087, 0.0002524)$\\
\SI{95}{\percent} & 4 & $(0.9999725, \num[scientific-notation=true]{3.44e-05})$ & $(0.9999737, \num[scientific-notation=true]{3.6e-05})$ & $(0.9998506, 0.0001942)$ & $(0.9998518, 0.0001958)$\\
\SI{99}{\percent} & 4 & $(0.9999783, \num[scientific-notation=true]{2.68e-05})$ & $(0.9999795, \num[scientific-notation=true]{2.83e-05})$ & $(0.9998756, 0.0001615)$ & $(0.9998768, 0.000163)$\\
\SI{99.9}{\percent} & 4 & $(0.9999833, \num[scientific-notation=true]{2.02e-05})$ & $(0.9999845, \num[scientific-notation=true]{2.18e-05})$ & $(0.9998889, 0.000144)$ & $(0.9998901, 0.0001456)$\\
\cmidrule(lr){1-6}
\SI{90}{\percent} & 5 & $(0.9998172, 0.0003232)$ & $(0.9998194, 0.0003273)$ & $(0.9994812, 0.0009246)$ & $(0.9994834, 0.0009286)$\\
\SI{95}{\percent} & 5 & $(0.9998259, 0.0003075)$ & $(0.9998282, 0.0003116)$ & $(0.9996371, 0.0006455)$ & $(0.9996394, 0.0006495)$\\
\SI{99}{\percent} & 5 & $(0.9998428, 0.0002774)$ & $(0.999845, 0.0002814)$ & $(0.9996703, 0.0005862)$ & $(0.9996725, 0.0005901)$\\
\SI{99.9}{\percent} & 5 & $(0.9998573, 0.0002517)$ & $(0.9998593, 0.0002553)$ & $(0.9996884, 0.000554)$ & $(0.9996904, 0.0005576)$\\
\cmidrule(lr){1-6}
\SI{90}{\percent} & 6 & $(0.9990169, 0.002336)$ & $(0.9990249, 0.0023554)$ & $(0.9978983, 0.0050219)$ & $(0.9979064, 0.0050413)$\\
\SI{95}{\percent} & 6 & $(0.9990368, 0.002288)$ & $(0.9990449, 0.0023074)$ & $(0.998714, 0.0030633)$ & $(0.998722, 0.0030827)$\\
\SI{99}{\percent} & 6 & $(0.9990736, 0.0021997)$ & $(0.9990817, 0.0022191)$ & $(0.998765, 0.0029407)$ & $(0.9987731, 0.0029601)$\\
\SI{99.9}{\percent} & 6 & $(0.9991069, 0.0021197)$ & $(0.999115, 0.0021392)$ & $(0.9987884, 0.0028845)$ & $(0.9987965, 0.0029039)$\\
\bottomrule
    \end{tabular}
\end{sidewaystable}


\begin{sidewaystable}
    \caption{Empirical confidence regions for time series of length $T = \num[scientific-notation=true]{1 e3}$ in the $H \times C$ plane obtained by the proposed methodology}
    \label{Tab:Regions1000}
    \centering
    \begin{tabular}{cccccc}
    \toprule
Confidence Region & $D$ & $(h_{v_1}, c_{v_1})$ & $(h_{v_2}, h_{v_2})$ & $(h_{v_3}, c_{v_3})$ & $(h_{v_4},c_{v_4})$\\ 
\midrule
\SI{90}{\percent} & 3 & $(0.9973334, 0.0025601)$ & $(0.9974047, 0.0026304)$ & $(0.9999497, 0)$ & $(1, \num[scientific-notation=true]{5.17e-05})$\\
\SI{95}{\percent} & 3 & $(0.9967311, 0.0031343)$ & $(0.9968219, 0.0032238)$ & $(0.9999398, 0)$ & $(1, \num[scientific-notation=true]{6.12e-05})$\\
\SI{99}{\percent} & 3 & $(0.9953009, 0.0045054)$ & $(0.9954349, 0.0046375)$ & $(0.9999203, 0)$ & $(1, \num[scientific-notation=true]{8.45e-05})$\\
\SI{99.9}{\percent} & 3 & $(0.9931825, 0.0065387)$ & $(0.9933704, 0.006724)$ & $(0.9998925, 0)$ & $(1, 0.0001104)$\\
\cmidrule(lr){1-6}
\SI{90}{\percent} & 4 & $(0.994364, 0.0081246)$ & $(0.9939234, 0.0075452)$ & $(0.9994791, 0.0013982)$ & $(0.9990385, 0.0008188)$\\
\SI{95}{\percent} & 4 & $(0.9937138, 0.0089796)$ & $(0.9932534, 0.0083741)$ & $(0.9991609, 0.0018166)$ & $(0.9987005, 0.0012111)$\\
\SI{99}{\percent} & 4 & $(0.9922575, 0.0108947)$ & $(0.9917308, 0.0102022)$ & $(0.9987924, 0.0023012)$ & $(0.9982658, 0.0016087)$\\
\SI{99.9}{\percent} & 4 & $(0.9902578, 0.0135243)$ & $(0.9897312, 0.0128318)$ & $(0.9985727, 0.0025901)$ & $(0.9980461, 0.0018976)$\\
\cmidrule(lr){1-6}
\SI{90}{\percent} & 5 & $(0.9811818, 0.0321294)$ & $(0.9827429, 0.0350291)$ & $(0.9919707, 0.0120896)$ & $(0.9935319, 0.0149893)$\\
\SI{95}{\percent} & 5 & $(0.9801289, 0.0340045)$ & $(0.9817117, 0.0369446)$ & $(0.9909376, 0.0139279)$ & $(0.9925204, 0.016868)$\\
\SI{99}{\percent} & 5 & $(0.977917, 0.0377295)$ & $(0.9796031, 0.0408613)$ & $(0.9898161, 0.0156277)$ & $(0.9915021, 0.0187595)$\\
\SI{99.9}{\percent} & 5 & $(0.9753326, 0.0425299)$ & $(0.9770187, 0.0456617)$ & $(0.9892599, 0.0166608)$ & $(0.9909459, 0.0197926)$\\
\cmidrule(lr){1-6}
\SI{90}{\percent} & 6 & $(0.9121895, 0.2201993)$ & $(0.9146048, 0.2260776)$ & $(0.9443868, 0.1418373)$ & $(0.9468021, 0.1477156)$\\
\SI{95}{\percent} & 6 & $(0.9105951, 0.2239294)$ & $(0.9130413, 0.2298829)$ & $(0.9419202, 0.1476904)$ & $(0.9443663, 0.1536439)$\\
\SI{99}{\percent} & 6 & $(0.9105951, 0.2239294)$ & $(0.9130413, 0.2298829)$ & $(0.9396577, 0.1531967)$ & $(0.9421039, 0.1591502)$\\
\SI{99.9}{\percent} & 6 & $(0.9077672, 0.2305874)$ & $(0.9102595, 0.2366531)$ & $(0.9383611, 0.1561279)$ & $(0.9408534, 0.1621937)$\\
\bottomrule
    \end{tabular}
\end{sidewaystable}
\section{Experimental Settings}

We evaluated the performance of the proposed method in relation to a large set of random sequences provided by state-of-the-art pseudo-random number generators.
In this section, we present the settings of the parameters that we use as a reference, the true random physical generators used to calculate the empirical distribution, and descriptive analysis of representative points in relation to the confidence regions.

\subsection{Parameters Settings and Dataset}

In the calculation of the ordinal symbol histogram, we employed the following factors in this study:
\begin{itemize}
\item Sequence length $T\in\mathcal T=\{10^3, \num[scientific-notation=true]{5 e4}\}$,
\item Embedding dimension $D\in\mathcal D=\{3, 4, 5, 6\}$,and
\item Time delay $\tau=\{1, 10, 30, 50\}$.
\end{itemize}

For the construction of the confidence regions presented, we use:
\begin{itemize}
    \item Set of $104596$ points in the $H \times C$ plane, referring to sequences of length $T = 1000$, for each combination of the factors $\mathcal T \times \mathcal D \times \mathcal{\tau}$, and
    \item  Another set of $2093$ points in the $H \times C$ plane, referring to sequences of length $T = 50000$, for each combination of factors $\mathcal T \times \mathcal D \times \mathcal{\tau}$.
\end{itemize}
Since the results involving the variation of the time delay parameter did not show a significant difference in repeated experiments, we don't consider it as a determinating factor during the execution of the algorithm.
On the other hand, we consider two determining variables during the generation of such sub-spaces: the embedding dimension and the length of the sequence.
%\todo[inline]{Justificar o porquê da ausência do tau nos experimentos}

The data generation and analyses were performed using the \texttt R platform \cite{Rmanual} v.~3.6.3.
We used the \texttt{ggplot2} library \cite{ggplot2Wickman} for generating the plots.

\subsection{True Random Numbers}

\todo[inline]{Gerador - Marcelo}

Random numbers are used in many fields, from gambling to criptography, aiming to guarantee a secure, realistic or unpredictable behavihor. 
Pseudo randomic results can be achieved by software in a deterministic way. 
But, some applications need actual random numbers (despite the somewhat elusive nature of actual randomness).
Randomness can be observed in unpredictable real world phenomena like cathodic radiation or atmospheric noise.
In this study we used two sources of real random numbers. 
The first is based on vacuum states to generate random quantum numbers described by \citeauthor{RNGVacuumStates}~(\citeyear{RNGVacuumStates}), the second one is based on atmospheric noise captured by a cheap radio receiver presented at \url{www.random.org}.

%Deixei as URLs pois não tenho o .bib ...

%The colored $k$-noise mentioned in the previous section, also known as ``correlated noise'', is such that its power spectrum has shape $f^{-k}$.
%When $k=0$, we have white noise, i.e., independent deviates.
%The larger $k$ is, the more correlated the observations are.
\begin{comment}

\begin{figure}
    \centering
    \includegraphics[width=\linewidth]{Figures/Points-PDF.png}
    \caption{White noise samples considered during the construction of the proposed confidence regions.}
    \label{fig:white-noise}
\end{figure}

\end{comment}
\section{Results}\label{Sec:Results}

\subsection{Descriptive analysis of empirical confidence regions}

The regions used as a reference in this work are obtained through true random sequences, where we extract the empirical distribution of white noises in the Entropy-Complexity plane.
In Fig.~\ref{fig:HC-PCA} we show the results obtained for $T = 50000$ in the scenarios of $D = 3$ and $D = 6$ in the new space defined by the PCA, together with the quantiles of $\SI{90}{\percent}$, $\SI{95}{\percent}$, $\SI{99}{\percent}$, and $\SI{99.9}{\percent}$.
We also show the projection of the $H \times C$ plane limits in this new representation space, as well as identifying the median of each data set, the latter being represented as the red dots present in the graphs.

\begin{figure}
	\centering
	\includegraphics[width=.45\linewidth]{Figures/HC-PCA-Trozos-D3N50k.png}
	\includegraphics[width=.45\linewidth]{Figures/HC-PCA-Trozos-D6N50k.png}
	\caption{Representation of truly random sequences with length $T = 50.000$ in the PCA space for $D = 3$ and $D = 6$, and the quantiles of $\SI{90}{\percent}$, $\SI{95}{\percent}$, $\SI{99}{\percent}$, and $\SI{99.9}{\percent}$.}
	\label{fig:HC-PCA}
\end{figure} 

As we can see in Fig.~\ref{fig:PCA-Hist} in the new representation space produced by the PCA, the data are not evenly distributed among the axes of the first main component, maintaining the character presented in the $H \times C$ plane, since such points tend to be concentrated close to the point $(1, 0)$.

\begin{figure}
    \centering
    \includegraphics[width=\linewidth]{PCA-hist-50k}
    \caption{Histogram of the PCA first component }
    \label{fig:PCA-Hist}
\end{figure}

\subsection{Testing White Noise in the confidence regions}

To analyze the efficiency of the confidence region calculated, we tested its applicability on a set of true random data generated physically not used by the algorithm during its construction. 
The results can be seen in Fig.~\ref{fig:RNG}.

For small series, $T = 1000$, and $D = 3$ we managed to maintain exactly $\SI{99}{\percent}$ of the data in the confidence region of the same value, and as the dimension increased reach $\SI{96}{\percent}$ of the points.
On the other hand, there was a very large loss of points located in the region with $\SI{95}{\percent}$ confidence as the dimension increased.
A reasonable explanation for this event is given in the choice of the parameter itself.
It is known by definition that $D! << N$, which does not happen for such a sample size, thus presenting many missing patterns that lead to an unrepresentative probability distribution.
For larger series, $T = 50000$, although we observed a small drop in the percentage of data present in the region with $\SI{99}{\percent}$ confidence, there was a significant increase in points in the region with $\SI{95}{\percent}$ confidence, showing between $\SI{90}{\percent}$ and $\SI{88}{\percent}$ of the points when we vary the embedding dimension.

\begin{figure}
    \centering
    \includegraphics[width=\linewidth]{Figures/RNG-1000.pdf}
    \includegraphics[width=\linewidth]{Figures/RNG-50000.pdf}
    \caption{Results of the analysis behavior of true random noises in the regions of confidence built.}
    \label{fig:RNG}
\end{figure}

\subsection{Analyzing Robustness to Correlation Structures}
% k=0 white
% k=1 pink
% k=2 brown
% T = 1000, T = 5 10^5
% D= 3, D = 6

Fig.~\ref{fig:AllSystems} shows the behavior of random time series with different levels of correlation (by means of the $f^{-k}$ model) in the $(h,c)$ plane.
Knowing that such plane can discriminate between different system dynamics, several studies in the literature have used this approach in the investigation of methods of identification and characterization of randomness.
Although this same strategy can be used to characterize different levels of correlation structures, in our case, we want to analyze the impact of injecting such dynamics into noise under the aspect of confidence regions.

For carrying out the experiment, we used an ''emblematic'' time series as a basis.
This series consists of the sample corresponding to the median of the $(h,c)$ points used to build the confidence regions, thus expressing a representative sample of the dataset.
Fig.~\ref{fig:correlation}a. illustrates, respectively, the effect of a white noise time series when adding correlation structures related to the $f^{-k}$ series for $k \in \{0, 1, 2, 3 \}$.
As we can observe in the plane as the correlation between the observations increases, that is, $k > 0$, the randomness decreases and the entropy presented decreases, informing the loss of its stochastic characteristic.

Fig.~\ref{fig:correlation}b. illustrates the degree of limit correlation structure that can be added in white noise to eliminate it from the regions of confidence, where the red dot represents the original "emblematic" series.
When we have $k = 0$, the features of the sequence have a small variation, corroborating the premise that series of noise $f^{-k} = 0$ have a minimum correlation measure, not significantly changing the dynamics of the system.

\begin{figure}
    \centering
	\includegraphics[width=.48\linewidth]{Figures/Correlation-Analysis-dotted.pdf}
    \includegraphics[width=.48\linewidth]{Figures/Correlation-Analysis-point.pdf}
    \caption{Correlation Structure Analysis}
    \label{fig:correlation}
\end{figure}

\subsection{Revisiting the White Noise Hypothesis in the Literature}

In this section, we evaluate the quality of the confidence regions preceded by HC-PCA with sequences of previously analyze generators in the literature with the Entropy-Complexity plane.
For this, we produced $100$ sequences of length $T = \num[scientific-notation=true]{5 e4}$ for each generator present in table~\ref{Tab:Literature} and calculated the respective p-values for each configuration of $D = \{3, 4, 5, 6\}$.
The results obtained can be seen in table~\ref{Tab:LiteratureComparations}, where we categorize the sets of sequences among those that did not reject (NR) the null hypothesis presented in the HC-PCA test and those that rejected (R).

Performing a comparative analysis of the results produced by tables~\ref{Tab:Literature} and~\ref{Tab:LiteratureComparations}, we can see that the proposed methodology of confidence regions can capture the random dynamics of the sequences produced by most of the analyzed generators well, although in our experiments we consider only short sequences.

However, two results deserve special attention: the sequences produced by the generators fBm and Combo RNG.
The first one, although it was not rejected by the analysis of~\cite{olivares2012contrasting}, presented low p-values, due to the characteristic point in the $H \times C$ plane where its sequences belong.
For the case analyzed in~\cite{olivares2012contrasting} with $D = 6$, we verified that the produced sequences present an average of $H = 0.997$, therefore they do not belong to the empirical $\SI{95}{\percent}$ confidence region produced by HC-PCA.
Analyzing the results produced by the Combo RNG, we can verify that only for $D = 3$ the proposed method can characterize the sequences produced as random.
This is due to the fact that by presenting higher dimension values, we were able to analyze a greater number of ordinal patterns, which may not be presented in their entirety in short sequences.

\begin{table}
    \caption{Results of the sequences generated by the main PRNGs in the literature. 
    The analyzed sequences have length $T=\num[scientific-notation = true]{5 e4}$.}
    \label{Tab:LiteratureComparations}
    \centering
    \begin{tabular}{cccc}
    \toprule
		Algorithm & 
		\multicolumn{1}{c}{$D$} & 
		$p$-value &
		HC-PCA\\
		\cmidrule(lr){1-1}
		\cmidrule(lr){2-2}
		\cmidrule(lr){3-3}
		\cmidrule(lr){4-4}
        MOT & 3 & 0.305 & NR\\
		& 4 & 0.572 & NR\\ 
		& 5 & 0.455 & NR\\ 
		& 6 & 0.508 & NR\\ 
		\cmidrule(lr){1-4}
        MWC & 3 & 0.501 & NR\\
        & 4 & 0.477 & NR\\ 
        & 5 & 0.496 & NR\\ 
        & 6 & 0.496 & NR\\ 
		\cmidrule(lr){1-4}
		COM & 3 & 0.123 & NR\\
		& 4 & 0.002 & R\\ 
		& 5 & \num[scientific-notation=true]{1.11 e-16} & R\\ 
		& 6 & \num[scientific-notation=true]{1.11 e-16} & R\\ 
		\cmidrule(lr){1-4}
		LEH & 3 & 0.531 & NR\\
		& 4 & 0.515 & NR\\ 
		& 5 & 0.495 & NR\\ 
		& 6 & 0.501 & NR\\ 
    \bottomrule
    \end{tabular}
    \begin{tabular}{|cccc}
    \toprule
		Algorithm & 
		\multicolumn{1}{c}{$D$} & 
		$p$-value &
		HC-PCA\\
		\cmidrule(lr){1-1}
		\cmidrule(lr){2-2}
		\cmidrule(lr){3-3}
		\cmidrule(lr){4-4}
		fGn & 3 & 0.521 & NR\\
		& 4 & 0.519 & NR\\ 
		& 5 & 0.498 & NR\\ 
		& 6 & 0.470 & NR\\
		\cmidrule(lr){1-4}
		fBm & 3 & \num[scientific-notation=true]{1.11 e-16} & R\\ 
		& 4 & \num[scientific-notation=true]{1.11 e-16} & R\\ 
		& 5 & \num[scientific-notation=true]{1.11 e-16} & R\\ 
		& 6 & \num[scientific-notation=true]{1.11 e-16} & R\\ 
		\cmidrule(lr){1-4}
		$f^{-k}$ & 3 & 0.482 & NR\\
		& 4 & 0.520 & NR\\ 
		& 5 & 0.513 & NR\\ 
		& 6 & 0.508 & NR\\
		\cmidrule(lr){1-4}
		LCG & 3 & 0.009 & R\\ 
		& 4 & \num[scientific-notation=true]{1.11 e-16} & R\\ 
		& 5 & \num[scientific-notation=true]{1.11 e-16} & R\\ 
		& 6 & \num[scientific-notation=true]{1.11 e-16} & R\\ 
    \bottomrule
    \end{tabular}
\end{table}

\begin{comment}
\begin{table}
    \caption{Results of literature PRNG references for time series of size $T = 50000$}
    \label{Tab:LiteratureComparations}
    \centering
    \begin{tabular}{c*{3}rcccc}
    \toprule
		Algorithm & 
		\multicolumn{1}{c}{$D$} & \multicolumn{1}{c}{\SI{95}{\percent}} & \multicolumn{1}{c}{\SI{99}{\percent}} & 
		$p$-value &
		HC-PCA & 
		Literature Results & 
		TestU01\\
		\cmidrule(lr){1-1}
		\cmidrule(lr){2-2}
		\cmidrule(lr){3-3}
		\cmidrule(lr){4-4}
		\cmidrule(lr){5-5}
		\cmidrule(lr){6-6}
		\cmidrule(lr){7-7}
		\cmidrule(lr){8-8}
        MOT & 3 & 1 & 1 & 0.305 & NR & NR & \\
		& 4 & 1 & 1 & 0.572  & & & \\ 
		& 5 & 1 & 1 & 0.455 & & & \\ 
		& 6 & 0.75 & 1 & 0.508 & & & \\ 
		\cmidrule(lr){1-8}
        MWC & 3 & 0.97 & 0.9625 & 0.501  & NR & NR & R\\
        & 4 & 0.97 & 0.9825 & 0.477 & & & \\ 
        & 5 & 0.9625 & 0.9625 & 0.496 & & & \\ 
        & 6 & 0.9675 & 0.9625 & 0.496  & & & \\ 
		\cmidrule(lr){1-8}
		COM & 3 & 0.25 & 0.5 & 0.123 & NR & NR & \\
		& 4 & 0 & 0 & 0.002 & R & & \\ 
		& 5 & 0 & 0 & \num[scientific-notation=true]{1.11 e-16} & & & \\ 
		& 6 & 0 & 0 & \num[scientific-notation=true]{1.11 e-16}  & & & \\ 
		\cmidrule(lr){1-8}
		LEH & 3 & 0.9825 & 0.98 & 0.531 & NR & NR & NR\\
		& 4 & 0.9675 & 0.9775 &  0.515  & & & \\ 
		& 5 & 0.9725 & 0.9775 &  0.495 & & & \\ 
		& 6 & 0.9775 & 0.9825 & 0.501 & & & \\ 
		\cmidrule(lr){1-8}
		fGn & 3 & 0.9722 & 0.9874 & 0.521 & NR & NR & \\
		& 4 & 0.9798 & 0.9773 & 0.519 & & & \\ 
		& 5 & 0.9646 & 0.9773 & 0.498 & & & \\ 
		& 6 & 0.9722 & 0.9545 & 0.470 & & & \\
		\cmidrule(lr){1-8}
		fBm & 3 & 0 & 0 & \num[scientific-notation=true]{1.11 e-16} & R & NR & \\ 
		& 4 & 0 & 0 & \num[scientific-notation=true]{1.11 e-16} & & & \\ 
		& 5 & 0 & 0 & \num[scientific-notation=true]{1.11 e-16} & & & \\ 
		& 6 & 0 & 0 & \num[scientific-notation=true]{1.11 e-16} & & & \\ 
		\cmidrule(lr){1-8}
		$f^{-k}$ & 3 & 0.9525 & 0.96 & 0.482 & NR & NR &\\
		& 4 & 0.975 & 0.9725 & 0.520 & & & \\ 
		& 5 & 0.98 & 0.9675 & 0.513 & & & \\ 
		& 6 & 0.9775 & 0.975 & 0.508 & & & \\
		\cmidrule(lr){1-8}
		LCG & 3 & 0.07 & 0.09 & 0.009 & R & R & R \\ 
		& 4 & 0 & 0 & \num[scientific-notation=true]{1.11 e-16} & & & \\ 
		& 5 & 0 & 0 & \num[scientific-notation=true]{1.11 e-16} & & & \\ 
		& 6 & 0 & 0 & \num[scientific-notation=true]{1.11 e-16} & & & \\ 
    \bottomrule
    \end{tabular}
\end{table}
\end{comment}

\begin{comment}
\begin{table}
	\centering
	\caption{Results of analysis of PRNGs with the proposed methodology}
	\label{tab:resultFinal}
	\begin{tabular}{c*{3}rccc}
		\toprule
		Algorithm & 
		\multicolumn{1}{c}{$D$} & 
		\multicolumn{1}{c}{\SI{95}{\percent}} & \multicolumn{1}{c}{\SI{99}{\percent}} &  
		$p$-value &
		HC-PCA &   
		TestU01\\
		\cmidrule(lr){1-1}
		\cmidrule(lr){2-2}
		\cmidrule(lr){3-3}
		\cmidrule(lr){4-4}
		\cmidrule(lr){5-5}
		\cmidrule(lr){6-6}
		\cmidrule(lr){7-7}
		Wichmann-Hill & 3 & 0.97 & 0.96 & 0.478 & NR & R\\ 
		& 4 & 0.9675 & 0.9675 & 0.513 & & \\ 
		& 5 & 0.9725 & 0.965 & 0.501 & & \\ 
		& 6 & 0.97 & 0.97 & 0.491 & & \\ 
		\cmidrule(lr){1-7}
		Super-Duper & 3 & 0.9675 & 0.975 & 0.496 & NR & R\\
		& 4 & 0.9775 & 0.9725 & 0.512 & &\\ 
		& 5 & 0.9675 & 0.985 & 0.525 & &\\ 
		& 6 & 0.9825 & 0.965 & 0.507 & &\\ 
		\cmidrule(lr){1-7}
		Knuth-TAOCP-2002 & 3 & 0.9775 & 0.9625 & 0.513 & NR & NR\\ 
		& 4 & 0.965 & 0.98 & 0.481 & &\\ 
		& 5 & 0.9625 & 0.9825 & 0.494 & &\\ 
		& 6 & 0.9725 & 0.96 & 0.479 & &\\  
		\cmidrule(lr){1-7}
		Knuth-TAOCP & 3 & 0.975 & 0.9775 & 0.480 & NR & NR\\
		& 4 & 0.965 & 0.9725 & 0.519 & &\\ 
		& 5 & 0.9875 & 0.985 & 0.531 & &\\ 
		& 6 & 0.9725 & 0.975 & 0.515 & &\\ 
		\cmidrule(lr){1-7}
		LEcuyer-CMRG & 3 & 0.98 & 0.98 & 0.505 & NR & NR\\ 
		& 4 & 0.9675 & 0.9725 & 0.490 & &\\ 
		& 5 & 0.965 & 0.9725 & 0.512 & &\\ 
		& 6 & 0.9625 & 0.97 & 0.524 & &\\ 
		\cmidrule(lr){1-7}
		Mersenne-Twister & 3 & 0.9725 & 0.97 & 0.506 & NR & NR\\ 
		& 4 & 0.98 & 0.97 & 0.499 & &\\ 
		& 5 & 0.9725 & 0.9775 & 0.524 & &\\ 
		& 6 & 0.975 & 0.9725 & 0.502 & &\\ 
		\cmidrule(lr){1-7}
		Randu & 3 & 0.2275 & 0.42 & 0.032 & R & R\\ 
		& 4 & 0 & 0 & \num[scientific-notation=true]{1.11 e-16} & &\\ 
		& 5 & 0 & 0 & \num[scientific-notation=true]{1.11 e-16} & &\\ 
		& 6 & 0 & 0 & \num[scientific-notation=true]{1.11 e-16} & &\\  
		\cmidrule(lr){1-7}
		pcg64 & 3 & 0.97 & 0.9675 & 0.510 & NR & NR\\
		& 4 & 0.97 & 0.97 & 0.503 & &\\ 
		& 5 & 0.9775 & 0.9625 & 0.508 & &\\ 
		& 6 & 0.9575 & 0.9575 & 0.502 & &\\ 
		\cmidrule(lr){1-7}
		Threefry & 3 & 0.9625 & 0.97 & 0.491 & NR & NR\\ 
		& 4 & 0.975 & 0.97 & 0.506 & &\\ 
		& 5 & 0.96 & 0.9625 & 0.506 & &\\ 
		& 6 & 0.965 & 0.955 & 0.473 & &\\ 
		\cmidrule(lr){1-7}
		Xoroshiro128+ & 3 & 0.975 & 0.9825 & 0.501 & NR & R\\ 
		& 4 & 0.995 & 0.9825 & 0.493 & &\\ 
		& 5 & 0.98 & 0.9725 & 0.503 & &\\ 
		& 6 & 0.9725 & 0.955 & 0.513 & &\\
		\cmidrule(lr){1-7}
		Xoshiro256+ & 3 & 0.98 & 0.98 & 0.502 & NR & NR\\ 
		& 4 & 0.9675 & 0.965 & 0.489 & &\\ 
		& 5 & 0.9525 & 0.96 & 0.492 & &\\ 
		& 6 & 0.9775 & 0.975 & 0.513 & &\\
		\bottomrule
	\end{tabular}
\end{table}
\end{comment}
\section{Conclusions}\label{Sec:Conclusions}

We present and evaluate a new method of building empirical confidence regions in the Entropy-Complexity plane for the analysis of white noise.
The following proposal consists of two stages:
(1)~the construction of empirical confidence regions obtained through the mapping of points in the latent space formed by the application of the principal components analysis under the reference points,
(2)~the application of a hypothesis test to measure the similarity of new sequences to the reference points.
Sequences of true random samples were used to calculate theses empirical regions.

Experiments with true random samples showed that the presented methodology can represent them with good performance.
Although the present work focuses on the study of short sequences, we were able to capture the random behavior of PRNGs already analyzed in the literature and we verified the robustness of our technique and the correlation structure present in the sequences.

\section{Source Code Availability}\label{Sec:code}

The text, source code, and data used in this study are available at the \textit{Confidence-Regions} repository \url{https://github.com/EduardaChagas/ConfidenceRegions}.



\section{Acknowledgements}\label{Sec:acknowledgements}
	
This work was partially funded by the Coordination for the Improvement of Higher Education Personnel (CAPES) and National Council for Scientific and Technological Development (CNPq).
\bibliography{references}
\appendix
\section{Empirical Confidence Regions}

As mentioned in the section~\ref{confidenceRegions}, it uses reference sequences for the construction of empirical confidence regions applied in this article. 
Below we present the points obtained in each of the configurations of size of the sequence and embedding dimension evaluated.

\begin{comment} 
\begin{table}
	\centering
	\caption{Results found it for samples of true random numbers}
	\label{tab:result1}
	\begin{tabular}{c*{3}rrr}
		\toprule
		Algorithm & \multicolumn{1}{c}{$N$} & \multicolumn{1}{c}{$D$} & \multicolumn{1}{c}{\SI{95}{\percent}} & \multicolumn{1}{c}{\SI{99}{\percent}} & $p$-value\\
		\cmidrule(lr){1-1}
		\cmidrule(lr){2-2}
		\cmidrule(lr){3-3}
		\cmidrule(lr){4-4}
		\cmidrule(lr){5-5}
		\cmidrule(lr){6-6}
		%True-Random & 1000 & 3 & $0.98$ & $1$ & \\
		%&  & 4 & $0.98$ & $0.96$ & \\
		%&  & 5 & $1$ & $0.94$ & \\
		%&  & 6 & $0.97$ & $0.87$ & \\
		%\cmidrule(lr){1-6} 
		True-Random & 50000 & 3 & $0.97$ & $0.96$ & $0.47856$\\
		& & 4 & $0.94$ & $0.95$ & $0.47555$\\ 
		& & 5 & $0.97$ & $0.96$ & $0.48512$\\ 
		& & 6 & $0.98$ & $0.99$ & $0.48058$\\
		\bottomrule
	\end{tabular}
\end{table}
\end{comment} 

\begin{sidewaystable}
    \caption{Empirical confidence regions for sequences of length $T = \num[scientific-notation=true]{5 e4}$ in the $H \times C$ plane}
    \label{Tab:Regions50k}
    \centering
    \begin{tabular}{cccccc}
    \toprule
    Confidence Regions & $D$ & $(h_{v_1}, c_{v_1})$ & $(h_{v_2}, h_{v_2})$ & $(h_{v_3}, c_{v_3})$ & $(h_{v_4},c_{v_4})$\\ 
    \midrule
    \SI{90}{\percent} & 3 & $(0.9999489, \num[scientific-notation=true]{5.06e-05})$ & $(0.9999487, \num[scientific-notation=true]{5.04e-05})$ & $(0.9999998, \num[scientific-notation=true]{4e-07})$ & $(0.9999996, \num[scientific-notation=true]{2e-07})$\\
    \SI{95}{\percent} & 3 & $(0.9999384, \num[scientific-notation=true]{6.11e-05})$ & $(0.9999382, \num[scientific-notation=true]{6.09e-05})$ & $(0.9999994, \num[scientific-notation=true]{9e-07})$ & $(0.9999991, \num[scientific-notation=true]{7e-07})$\\
    \SI{99}{\percent} & 3 & $(0.9999079, \num[scientific-notation=true]{9.11e-05})$ & $(0.9999077, \num[scientific-notation=true]{9.09e-05})$ & $(0.9999982, \num[scientific-notation=true]{2e-06})$ & $(0.999998, \num[scientific-notation=true]{1.8e-06})$\\
    \SI{99.9}{\percent} & 3 & $(0.9998625, 0.0001361)$ & $(0.9998622, 0.0001358)$ & $(0.9999973, \num[scientific-notation=true]{3e-06})$ & $(0.999997, \num[scientific-notation=true]{2.7e-06})$\\
    \cmidrule(lr){1-6}
    \SI{90}{\percent} & 4 & $(0.9999684, \num[scientific-notation=true]{3.98e-05})$ & $(0.9999696, \num[scientific-notation=true]{4.13e-05})$ & $(0.9998075, 0.0002508)$ & $(0.9998087, 0.0002524)$\\
    \SI{95}{\percent} & 4 & $(0.9999725, \num[scientific-notation=true]{3.44e-05})$ & $(0.9999737, \num[scientific-notation=true]{3.6e-05})$ & $(0.9998506, 0.0001942)$ & $(0.9998518, 0.0001958)$\\
    \SI{99}{\percent} & 4 & $(0.9999783, \num[scientific-notation=true]{2.68e-05})$ & $(0.9999795, \num[scientific-notation=true]{2.83e-05})$ & $(0.9998756, 0.0001615)$ & $(0.9998768, 0.000163)$\\
    \SI{99.9}{\percent} & 4 & $(0.9999833, \num[scientific-notation=true]{2.02e-05})$ & $(0.9999845, \num[scientific-notation=true]{2.18e-05})$ & $(0.9998889, 0.000144)$ & $(0.9998901, 0.0001456)$\\
    \cmidrule(lr){1-6}
    \SI{90}{\percent} & 5 & $(0.9998172, 0.0003232)$ & $(0.9998194, 0.0003273)$ & $(0.9994812, 0.0009246)$ & $(0.9994834, 0.0009286)$\\
    \SI{95}{\percent} & 5 & $(0.9998259, 0.0003075)$ & $(0.9998282, 0.0003116)$ & $(0.9996371, 0.0006455)$ & $(0.9996394, 0.0006495)$\\
    \SI{99}{\percent} & 5 & $(0.9998428, 0.0002774)$ & $(0.999845, 0.0002814)$ & $(0.9996703, 0.0005862)$ & $(0.9996725, 0.0005901)$\\
    \SI{99.9}{\percent} & 5 & $(0.9998573, 0.0002517)$ & $(0.9998593, 0.0002553)$ & $(0.9996884, 0.000554)$ & $(0.9996904, 0.0005576)$\\
    \cmidrule(lr){1-6}
    \SI{90}{\percent} & 6 & $(0.9990169, 0.002336)$ & $(0.9990249, 0.0023554)$ & $(0.9978983, 0.0050219)$ & $(0.9979064, 0.0050413)$\\
    \SI{95}{\percent} & 6 & $(0.9990368, 0.002288)$ & $(0.9990449, 0.0023074)$ & $(0.998714, 0.0030633)$ & $(0.998722, 0.0030827)$\\
    \SI{99}{\percent} & 6 & $(0.9990736, 0.0021997)$ & $(0.9990817, 0.0022191)$ & $(0.998765, 0.0029407)$ & $(0.9987731, 0.0029601)$\\
    \SI{99.9}{\percent} & 6 & $(0.9991069, 0.0021197)$ & $(0.999115, 0.0021392)$ & $(0.9987884, 0.0028845)$ & $(0.9987965, 0.0029039)$\\
    %\bottomrule
    \end{tabular}
    \quad
%\end{sidewaystable}
%\begin{sidewaystable}
    \caption{Empirical confidence regions for sequences of length $T = \num[scientific-notation=true]{1 e3}$ in the $H \times C$ plane}
%    \label{Tab:Regions1000}
%    \centering
    \begin{tabular}{cccccc}
    \toprule
Confidence Region & $D$ & $(h_{v_1}, c_{v_1})$ & $(h_{v_2}, h_{v_2})$ & $(h_{v_3}, c_{v_3})$ & $(h_{v_4},c_{v_4})$\\ 
\midrule
\SI{90}{\percent} & 3 & $(0.9973334, 0.0025601)$ & $(0.9974047, 0.0026304)$ & $(0.9999497, 0)$ & $(1, \num[scientific-notation=true]{5.17e-05})$\\
\SI{95}{\percent} & 3 & $(0.9967311, 0.0031343)$ & $(0.9968219, 0.0032238)$ & $(0.9999398, 0)$ & $(1, \num[scientific-notation=true]{6.12e-05})$\\
\SI{99}{\percent} & 3 & $(0.9953009, 0.0045054)$ & $(0.9954349, 0.0046375)$ & $(0.9999203, 0)$ & $(1, \num[scientific-notation=true]{8.45e-05})$\\
\SI{99.9}{\percent} & 3 & $(0.9931825, 0.0065387)$ & $(0.9933704, 0.006724)$ & $(0.9998925, 0)$ & $(1, 0.0001104)$\\
\cmidrule(lr){1-6}
\SI{90}{\percent} & 4 & $(0.994364, 0.0081246)$ & $(0.9939234, 0.0075452)$ & $(0.9994791, 0.0013982)$ & $(0.9990385, 0.0008188)$\\
\SI{95}{\percent} & 4 & $(0.9937138, 0.0089796)$ & $(0.9932534, 0.0083741)$ & $(0.9991609, 0.0018166)$ & $(0.9987005, 0.0012111)$\\
\SI{99}{\percent} & 4 & $(0.9922575, 0.0108947)$ & $(0.9917308, 0.0102022)$ & $(0.9987924, 0.0023012)$ & $(0.9982658, 0.0016087)$\\
\SI{99.9}{\percent} & 4 & $(0.9902578, 0.0135243)$ & $(0.9897312, 0.0128318)$ & $(0.9985727, 0.0025901)$ & $(0.9980461, 0.0018976)$\\
\cmidrule(lr){1-6}
\SI{90}{\percent} & 5 & $(0.9811818, 0.0321294)$ & $(0.9827429, 0.0350291)$ & $(0.9919707, 0.0120896)$ & $(0.9935319, 0.0149893)$\\
\SI{95}{\percent} & 5 & $(0.9801289, 0.0340045)$ & $(0.9817117, 0.0369446)$ & $(0.9909376, 0.0139279)$ & $(0.9925204, 0.016868)$\\
\SI{99}{\percent} & 5 & $(0.977917, 0.0377295)$ & $(0.9796031, 0.0408613)$ & $(0.9898161, 0.0156277)$ & $(0.9915021, 0.0187595)$\\
\SI{99.9}{\percent} & 5 & $(0.9753326, 0.0425299)$ & $(0.9770187, 0.0456617)$ & $(0.9892599, 0.0166608)$ & $(0.9909459, 0.0197926)$\\
\cmidrule(lr){1-6}
\SI{90}{\percent} & 6 & $(0.9121895, 0.2201993)$ & $(0.9146048, 0.2260776)$ & $(0.9443868, 0.1418373)$ & $(0.9468021, 0.1477156)$\\
\SI{95}{\percent} & 6 & $(0.9105951, 0.2239294)$ & $(0.9130413, 0.2298829)$ & $(0.9419202, 0.1476904)$ & $(0.9443663, 0.1536439)$\\
\SI{99}{\percent} & 6 & $(0.9105951, 0.2239294)$ & $(0.9130413, 0.2298829)$ & $(0.9396577, 0.1531967)$ & $(0.9421039, 0.1591502)$\\
\SI{99.9}{\percent} & 6 & $(0.9077672, 0.2305874)$ & $(0.9102595, 0.2366531)$ & $(0.9383611, 0.1561279)$ & $(0.9408534, 0.1621937)$\\
\bottomrule
    \end{tabular}
\end{sidewaystable}

\end{document}

