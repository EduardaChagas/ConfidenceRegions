\section{Introduction}\label{Sec:Intro}

Time Series carry valuable information about the system which produces the data.
Their analysis is usually based on two approaches \cite{TimeSeriesAnalysisCryerChan}: in the (natural) time and transformed domains (for instance, frequency and wavelet).

In the context of time-domain analysis, a new methodology was proposed by \citeauthor{Bandt2002Permutation}~\citeyear{Bandt2002Permutation}.
Its approach is non-parametric and based on descriptors of Information Theory.
Through this, the time series is transformed into ordinal patterns, with which a histogram is formed.
Using such patterns, the resulting distribution becomes less sensitive to outliers and, as it does not depend on any model, can be applied to a variety of situations.

The Bandt-Pompe methodology and its variants have been used successfully in the analysis of many types of dynamics, receiving so far more than \num{2500} citations, according to the Journal of Citation Records.
We found works using such an approach in multiple areas of scientific knowledge such as, for example,
the study of electroencephalography signals using wavelet decomposition~\cite{EEGAnalysisWaveletInformationTools},
characterization of household appliances through their energy consumption~\cite{CharacterizationElectricLoadInformationTheoryQuantifiers},
 online signature classification and verification~\cite{ClassificationVerificationOnlineHandwrittenSignatures}
 %, and characterization of SAR image textures~\cite{DudaLAGIRS}
 .

Each time series is described by a point in the range of $\mathbbm R^2$, the entropy complexity plane.
Two points are well known in this plane: those of white noise and a completely deterministic sequence.
Through these references, we can characterize time series according to the dynamics of its generating process.
Based on this premise, studies with different applications managed to obtain relevant results from time series through information on the nature of the data provided by the  $H \times C$ plane.
Examples include the work of \citeauthor{ComplexNetworksEvolution}~\citeyear{ComplexNetworksEvolution} analyzed the evolution of dynamic networks by trajectories in the $H \times C$ plane,
\citeauthor{InformationTheoryPerspectiveNetworkRobustness}~\citeyear{InformationTheoryPerspectiveNetworkRobustness} checked the effect of attacks to complex networks by the displacement of their points in the $H \times C$ plane,
\citeauthor{CharacterizationVehicleBehaviorInformationTheory}~\citeyear{CharacterizationVehicleBehaviorInformationTheory} described the behavior of vehicles as a function of the topology of cities, and
\citeauthor{StructuralChangesDataCommunicationWSN}~\citeyear{StructuralChangesDataCommunicationWSN} employ Information-Theoretic measures to describe the evolution of wireless sensors networks.

Many results in this regard have been provided through the analysis and study of chaotic components.
We can cite as highlights:
\citeauthor{GeneralizedStatisticalComplexityMeasuresGeometricalAnalyticalProperties}~\citeyear{GeneralizedStatisticalComplexityMeasuresGeometricalAnalyticalProperties} analyzed the logistic chaotic map and discuss the boundaries of the $H \times C$ plane.
\citeauthor{De_Micco_2009}~\citeyear{De_Micco_2009} studied chaotic components in pseudo-random number generators.
\citeauthor{DistinguishingNoiseFromChaos}~\citeyear{DistinguishingNoiseFromChaos}  tackled the often hard problem of distinguishing chaos and noise.
\citeauthor{DistinguishingChaoticStochasticDynamicsTimeSeriesMultiscaleSymbolicApproach}~\citeyear{DistinguishingChaoticStochasticDynamicsTimeSeriesMultiscaleSymbolicApproach} used a multi-scale approach to analyze the interplay between chaotic and stochastic dynamics.

With the knowledge of the expected variability of such points, according to the underlying dynamics, we can make hypothesis tests for a wide variety of models.
Some preliminary results can already be found in the literature.
\citeauthor{Larrondo2006Random}~\citeyear{Larrondo2006Random} showed that the Entropy-Complexity plane ($H\times C$) is a good indicator of the results of Diehard tests for pseudorandom number generators.
\citeauthor{De_Micco_2008}~\citeyear{De_Micco_2008} assessed ways of improving pseudorandom sequences by their representation in this plane.
\citeauthor{LiborInvisibleHand}~\citeyear{LiborInvisibleHand} identified spurious interventions in the Libor market using the $H\times C$ plane representation.

This paper advances the state-of-the-art by providing confidence regions for points in the $H\times C$ plane for a diversity of situations of interest.
%Such regions are obtained using a Monte Carlo approach.
The input is a sequence of actual random observations generated by a physical procedure, and these regions are obtained by the application of PCA in space of Entropy-Complexity Plane.
We emphasize that this study aims at providing confidence regions for sequences of the size that usually appear in the literature, not tests for randomness of PRNGs.


The paper is structured as follows: Section~\ref{Sec:Intro} introduces the elements of the study (the Bandt and Pompe methodology, the random deviates, and model).
The confidence regions are presented in Section~\ref{Sec:Results} with some analysis of this application, and the conclusions are discussed in Section~\ref{Sec:Conclusions}.
