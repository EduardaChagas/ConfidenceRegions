\section{Introduction}\label{Sec:Intro}

Time Series carry valuable information about the system which produces the data.
Their analysis is usually based on two approaches \citep{TimeSeriesAnalysisCryerChan}: 
in the (natural) time and transformed domains (for instance, frequency and wavelet).
In the context of time-domain analysis, \citet{PermutationEntropyBandtPompe} proposed a new methodology.
Such an approach is non-parametric and based on descriptors of Information Theory.
The time series is transformed into ordinal patterns, with which a histogram is formed.
The resulting distribution is less sensitive to outliers than the original data, and the histogram does not depend on any model and. 
Thus, the approach can be applied to a variety of situations.

The Bandt-Pompe methodology and its variants have been used successfully in the analysis of many types of dynamics, receiving so far more than \num{1800} citations, according to the Web of Science.
We found works using this approach in several areas of scientific knowledge such as, for example:
%
distinguishing noise from chaos~\citep{rosso2007distinguishing};
%
the study of electroencephalography signals using wavelet decomposition~\citep{baravalle2018discriminating,baravalle2018rhythmic};
%
description of El Niño/Southern Oscillation during the Holocene~\citep{saco2010entropy};
%
the characterization of household appliances through their energy consumption~\citep{CharacterizationElectricLoadInformationTheoryQuantifiers};
%
detecting and quantifying stochastic and coherence resonance~\citep{rosso2009detectinga, rosso2009detectingb};
%
analysis and characterization of economic time series, e.g., stock market, sovereign bonds, credit rating, commodities, and cryptocurrencies~\citep{zunino2010complexity, zunino2012efficiency, bariviera2013efficiency, bariviera2018analysis, Araujo2019permutation};
%
online signature classification and verification~\citep{ClassificationVerificationOnlineHandwrittenSignatures}.
\cite{InformationTheoryPerspectiveNetworkRobustness} verified the effect of attacks on complex networks by the displacement of points in the $H \times C$ plane.
\citet{CharacterizationVehicleBehaviorInformationTheory} described vehicles' behavior depending on the topology of cities, and
\citet{Chagas2020Characterization} succeeded in expanding the use of such techniques for analyzing textured images corrupted by speckle noise.
\citet{LiborInvisibleHand} identified spurious interventions in the Libor market using the $H\times C$ plane representation.
\citet{echegoyen2020permutation} were able to discriminate between individuals with mild cognitive impairment from those diagnosed with Alzheimer's disease using magnetoencephalography recordings.


With the Bandt and Pompe methodology, a time series is described by a point in a manifold of $\mathbbm R^2$: the Entropy-Complexity plane $H\times C$.
There are two well-known points in this plane: those of white noise and a completely deterministic sequence.
Although the boundaries of the $H\times C$ plane are well-defined, its intrinsic topology's complete characterization is an open problem.
In particular, the joint distribution of points in this plane under typical time series types, which would serve to build test statistics, is unknown.

The \num{1820} citations received by the seminal paper appeared in \num{684} journals indexed by the Web of Science.
These journals belong to \num{127} categories, spanning from Multidisciplinary Physics (\SI{24}{\percent} of the publications) to Zoology (only one of the of citing articles).
There are \num{17} citing articles from journals that belong to the Statistics \& Probability category.
Five of these articles appeared in \textit{Stochastic Environmental Research and Risk Assessment}, 
two in the \textit{Journal of Time Series Analysis}, 
and each of the remaining ten appeared in a different journal.
Most of these articles relate successful applications of the Bandt and Pompe methodology, except \cite{OrdinalPatternProbabilities} that obtained the sample entropy's properties under zero-mean Gaussian processes.
It is also noteworthy that, in this category of publications, \cite{DistributionsofOrderPatternsofIntervalMaps} provided a formal and more general proof of the structure of the boundary of the $H\times C$ manifold than that obtained by \cite{martin2006generalized}.
The lack of attention that the Bandt and Pompe approach has received by the Probability \& Statistics community confirms that it is a fertile research avenue waiting to explore.

Several works have used deterministic and pseudorandom sequences to understand the properties of the points they produce in the $H\times C$ plane.
\cite{martin2006generalized} analyzed the chaotic logistic map and discussed the boundaries of the $H \times C$ plane.
\cite{De_Micco_2009} studied chaotic components in pseudorandom number generators.
\cite{DistinguishingNoiseFromChaos}  tackled the often hard problem of distinguishing chaos from noise.
\cite{DistinguishingChaoticStochasticDynamicsTimeSeriesMultiscaleSymbolicApproach} used a multi-scale approach to analyze the interplay between chaotic and stochastic dynamics.

With the knowledge of the expected variability of such points, according to the underlying dynamics, we can make hypothesis tests for a wide variety of models.
Results in this direction can be found in the literature.
\cite{RandomNumberGeneratorsCausality} showed that the Entropy-Complexity plane ($H\times C$) is a good indicator of Diehard tests' results on pseudorandom number generators.
\cite{De_Micco_2008} assessed ways of improving pseudorandom sequences by their representation in this plane.

Motivated by previous works, in this paper, we advance the state-of-the-art providing the first test for white noise points in the $H\times C$ plane.
In this proposal, the input is a sequence of true random observations generated by a physical-based procedure. 
We obtain the confidence regions by performing an orthogonal projection of the data onto the space of principal components, thus eliminating the restrictions imposed by the bounded space of the Entropy-Complexity plane.
Our contributions can be summarized as follows:
\begin{itemize}
	\item We provide the first contribution in constructing a test in the Entropy-Complexity Plane: we provide confidence regions and $p$-values.
	\item We evaluate this test's size by analyzing random sequences generated by physical procedures and pseudorandom generators (PRNGs).
	\item We verify the test's power contrasting correlated noise time series.
\end{itemize}

The rest of this paper is structured as follows. 
Section~\ref{Sec:BP} introduces the elements of the study:
Section~\ref{Sec:BPMethodology} details the methodology,
Section~\ref{Sec:HCPlane} describes the Entropy-Complexity plane,
and Section~\ref{Sec:BPApplications} discusses applications of this approach.
%
Section~\ref{Sec:BPMethodology} describes our methodology:
Section~\ref{Sec:OverallFramework} gives the overall framework.
Section~\ref{Sec:TRNG} describes how we obtained true white noise random sequences from physical devices,
Section~\ref{Sec:Parameters} lists the parameters of relevance for the study, and
Section~\ref{confidenceRegions} describes how we obtained the confidence intervals and $p$-values.
%
Section~\ref{Sec:Results} assesses the proposed test by verifying its size and power, and its application to well-known pseudorandom number generators.
Section~\ref{Sec:Conclusions} concludes the paper with a discussion of these results.
